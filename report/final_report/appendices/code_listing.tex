\begin{landscape}
\begin{multicols}{2}
\chapter{Code Listing}

In the following section we include the code that implements the unsupervised deep learning models used.

\begingroup

\lstset{
    frame=none,
    breaklines=true,
    breakautoindent=true,
    postbreak={},
    upquote=true
}

\renewcommand{\thesubsection}{\arabic{subsection}}

\renewcommand{\addcontentsline}[3]{}% Do nothing

\subsection{Autoencoder}

\begin{lstlisting}[language=Python]
"""
Implements a simple stacked autoencoder.

Hyperparameters to tune:
------------------------
- Learning rate
- Activation function (sigmoid, ReLU, atan)
- Amount of neurons in each layer
- Learning function (GradientDescentOptimizer, RMSProp, AdamOptimizer)
- Batch size
"""
"""
Implements a simple stacked autoencoder.

Hyperparameters to tune:
------------------------
- Learning rate
- Activation function (sigmoid, ReLU, atan)
- Amount of neurons in each layer
- Learning function (GradientDescentOptimizer, RMSProp, AdamOptimizer)
- Batch size
"""
import numpy as np
import tensorflow as tf

from sys import stdout, path
from os import path as ospath

from sklearn.preprocessing import MinMaxScaler

path.append(ospath.dirname(ospath.dirname(ospath.abspath(__file__))))
import helpers

class AutoEncoder():
    """
    Implements an autoencoder that tries to learn a representation for web page traces.

    Atrributes:
        - activation_func is a tensorflow function, representing the activation function used.
            *(Often found in `tf.nn`)*
        - encoder is a computation, representing the encoder layers
        - decoder is another computtational graph, representing the decoder layers
        - loss is the operation for the mean squared error (MSE)
        - train_op is the train operation (`RMSProp`)
        - layers is a list of integerrs, determining the amount of layers and their size
        - is_training is a boolean representing whether you are training the autoencoder or not *(used in the batch_norm layer)*.
        - batch_size
        - learning_rate
    """

    def __init__(self, layers, batch_size, activation_func=tf.nn.sigmoid, saved_graph=None, sess=None, learning_rate=0.0001, batch_norm=False):
        """
        @param layers is a list of integers, determining the amount of layers and their size
            starting with the input size
        """
        if len(layers) < 2:
            print("Amount of layers must be greater than 1")
            exit(0)

        self.batch_size = batch_size
        self.learning_rate = learning_rate
        self.activation_func = activation_func
        self.batch_norm = batch_norm

        self.is_training = True

        # Use this in data preprocessing
        self.layers = layers

        self._make_graph(layers)

        if saved_graph is not None and sess is not None:
            self.import_from_file(sess, saved_graph)

    def _make_graph(self, layers):
        """
        Constructs the computational graph

        @param layers is a list of integers, determining the size of the layers
        """
        self._init_placeholders(layers[0])

        self.encoder = self._init_encoder(layers)
        self.decoder = self._init_decoder(layers)

        self._init_train()

    def _init_placeholders(self, first_layer):
        """
        The main placeholders for input and output data
        """
        self.encoder_inputs = tf.placeholder(tf.float32, [self.batch_size, first_layer])

        # We could technically use the same value as encoder_inputs but we do not
        # for future possible extensions
        self.decoder_targets = tf.placeholder(tf.float32, [self.batch_size, first_layer])


    def _get_layer(self, layer_input, size_last_layer, size_current_layer):
        """
        Returns a layer with a batch normalized input, depending on the `batch_norm flag`

        @param layer_input is the value used as an input to the layer.
        @param size_last_layer is the size of the last layer (used in weight) or the size of the input
        @param size_current_layer is the size of the current layer (used in weight and bias)
        """
        weight = tf.Variable(tf.random_normal([size_last_layer, size_current_layer]))
        bias = tf.Variable(tf.random_normal([size_current_layer]))

        if not self.batch_norm:
            return self.activation_func(tf.add(tf.matmul(layer_input, weight), bias))


        layer_input = tf.contrib.layers.batch_norm(layer_input,
                                                   center=True, scale=True,
                                                   is_training=self.is_training,
                                                   scope='bn{}-{}'.format(size_last_layer, size_current_layer))

        return self.activation_func(tf.add(tf.matmul(layer_input, weight), bias))

    def _init_encoder(self, layers):
        """
        Creates the layers of the decoder and returns the last layer.
        """
        previous_layer = None

        # We don't want to enumerate over the last one
        for i in range(len(layers) - 1):
            current_layer = None
            if previous_layer is None:
                current_layer = self._get_layer(self.encoder_inputs, layers[i], layers[i + 1])
            else:
                current_layer = self._get_layer(previous_layer, layers[i], layers[i + 1])

            previous_layer = current_layer

        # Will be the last layer
        return previous_layer

    def _init_decoder(self, layers):
        """
        Creates the decoder graph and returns the last layer
        """
        previous_layer = None

        # We don't want to enumerate over the last one
        for i in range(len(layers) - 1, 0, -1):
            current_layer = None
            if previous_layer is None:
                current_layer = self._get_layer(self.encoder, layers[i], layers[i - 1])
            else:
                current_layer = self._get_layer(previous_layer, layers[i], layers[i - 1])

            previous_layer = current_layer

        # Will be the last layer
        return previous_layer

    def _init_train(self):
        """
        Create the train operation
        """
        self.loss = tf.reduce_sum(tf.square(self.decoder_targets - self.decoder))

        # Which optimizer to use? `GradientDescentOptimizer`, `AdamOptimizer` or `RMSProp`?
        self.train_op = tf.train.AdamOptimizer(self.learning_rate).minimize(self.loss)

    def _init_batch_norm(self):
        """
        Adds a batch normalization layer.
        """
        self.decoder = tf.contrib.layers.batch_norm(self.decoder,
                                                    center=True, scale=True,
                                                    is_training=self.is_training,
                                                    scope='bn')

    def set_is_training(is_training):
        """
        Sets the `is_training` class variable, used in th batch normalization layer.
        If `batch_norm == False`, this does not make a difference but if it is true, this variable should be set to false after training.
        """
        self.is_training = is_training

    def _process_trace(self, trace, n):
        """
        Cuts the traces after `n` steps or pads them such that they are of length `n`.
        """
        features = []

        for packet in trace:
            # Either positive or negative depending on whether its incoming or outgoing.
            features.append(packet[0] * packet[1])

            if len(features) == n:
                break

        for i in range(len(features), n):
            features.append(0)

        return features


    def next_batch(self, batches, in_memory):
        """
        Returns the next batch in some fixed-length representation.
        Currently we use Panchenko et al.'s cumulative traces

        @param batches an iterator with all of the batches (
            if in_memory == True:
                in batch-major form without padding
            else:
                A list of paths to the files
        )
        @param in_memory is a boolean value

        @return if in_memory is False, returns a tuple of (dict, [paths]) where paths is a list of paths for each batch
            else it returns a dict for training
        """
        batch = next(batches)
        data_batch = batch

        if not in_memory:
            data_batch = [helpers.read_cell_file(path) for path in batch]

        data_batch = [self._process_trace(trace, self.layers[0]) for trace in data_batch]

        min_max_scaler = MinMaxScaler()
        data_batch = min_max_scaler.fit_transform(data_batch)

        encoder_inputs_ = data_batch
        decoder_targets_ = data_batch

        train_dict = {
            self.encoder_inputs: encoder_inputs_,
            self.decoder_targets: decoder_targets_,
        }

        if not in_memory:
            return (train_dict, batch)
        return train_dict

    def save(self, sess, file_name):
        """
        Save the model in a file

        @param sess is the session
        @param file_name is the file name without the extension
        """
        saver = tf.train.Saver()
        saver.save(sess, file_name)

    def import_from_file(self, sess, file_name):
        """
        Imports the graph from a file

        @param sess is the session
        @param file_name is a string that represents the file name
            without the extension
        """

        # Get the graph
        saver = tf.train.Saver()

        # Restore the variables
        saver.restore(sess, file_name)



def train_on_copy_task(sess, model, data,
                       batch_size=100,
                       max_batches=None,
                       batches_in_epoch=1000,
                       verbose=False):
    """
    Train the `AutoEncoder` on a copy task

    @param sess is a tensorflow session
    @param model is the autoencoder model
    @param data is the data (in batch-major form and not padded or a list of files (depending on `in_memory`))
    """
    batches = helpers.get_batches(data, batch_size=batch_size)

    loss_track = []

    batches_in_data = len(data) // batch_size
    if max_batches is None or batches_in_data < max_batches:
        max_batches = batches_in_data - 1

    try:
        for batch in range(max_batches):
            print("Batch {}/{}".format(batch, max_batches))
            fd, _ = model.next_batch(batches, False)
            _, l = sess.run([model.train_op, model.loss], fd)

            loss_track.append(l)

            if batch == 0 or batch % batches_in_epoch == 0:
                model.save(sess, 'autoencoder_model')
                helpers.save_object(loss_track, 'loss_track.pkl')

                if verbose:
                    stdout.write('  minibatch loss: {}\n'.format(sess.run(model.loss, fd)))
                    predict_ = sess.run(model.decoder_outputs, fd)
                    for i, (inp, pred) in enumerate(zip(fd[model.encoder_inputs].swapaxes(0, 1), predict_.swapaxes(0, 1))):
                        stdout.write('  sample {}:\n'.format(i + 1))
                        stdout.write('    input     > {}\n'.format(inp))
                        stdout.write('    predicted > {}\n'.format(pred))
                        if i >= 0:
                            break
                    stdout.write('\n')

    except KeyboardInterrupt:
        stdout.write('training interrupted')
        model.save(sess, 'autoencoder_model')
        exit(0)

    model.save(sess, 'autoencoder_model')
    helpers.save_object(loss_track, 'loss_track.pkl')

    return loss_track

def get_vector_representations(sess, model, data, save_dir,
                       batch_size=100,
                       max_batches=None,
                       batches_in_epoch=1000,
                       extension=".cell"):
    """
    Given a trained model, gets a vector representation for the traces in batch

    @param sess is a tensorflow session
    @param model is the autoencoder model
    @param data is the data (in batch-major form and not padded or a list of files (depending on `in_memory`))
    """
    batches = helpers.get_batches(data, batch_size=batch_size)

    batches_in_data = len(data) // batch_size
    if max_batches is None or batches_in_data < max_batches:
        max_batches = batches_in_data - 1

    try:
        for batch in range(max_batches):
            print("Batch {}/{}".format(batch, max_batches))
            fd, paths = model.next_batch(batches, False)
            l = sess.run(model.encoder, fd)

            file_names = [helpers.extract_filename_from_path(path, extension) for path in paths]

            for file_name, features in zip(file_names, list(l)):
                helpers.write_to_file(features, save_dir, file_name, new_extension=".cellf")

    except KeyboardInterrupt:
        stdout.write('Interrupted')
        exit(0)

    return results

\end{lstlisting}

\subsection{Sequence-to-sequence model}

\begin{lstlisting}[language=Python]
"""
This file implements a RNN encoder-decoder model (also known as sequence-to-sequence models).

We made the choice not to implement an attention mechanism (which means that the decoder is allowed to have a 'peak' at the input).
The reason why is because we are not trying to maximize the output of the decoder but instead the feature selection process.
(http://suriyadeepan.github.io/2016-06-28-easy-seq2seq/)

We will use batch-major rather than time-major even though time-major is slightly more efficient
since it makes the feature extraction process a lot easier.

We will not be using bucketing because traces of the same webpage will have the same length.
Therefore every batch, we will most likely be training the seq2seq model on one webpage

! Does encoder share weights with decoder or not (Less computation vs natural (https://arxiv.org/pdf/1409.3215.pdf))
! Reverse traces (https://arxiv.org/pdf/1409.3215.pdf)

Hyperparameters to tune:
------------------------
- Learning rate
- Which cell to use (GRU vs LSTM) or a deep RNN architecture using `MultiRNNCell`
- Reversing traces
- Bidirectional encoder
- Other objective functions (such as MSE,...)
- Amount of encoder and decoder hidden states
"""
import numpy as np
import tensorflow as tf

from sys import stdout, path
from os import path as ospath

from tensorflow.contrib.rnn import LSTMStateTuple

path.append(ospath.dirname(ospath.dirname(ospath.abspath(__file__))))
import helpers


class Seq2SeqModel():
    """
    Implements a sequence to sequence model for real values

    Attributes:
        - encoder_cell is the cell that will be used for encoding
            (Should be part of `tf.nn.rnn_cell`)
        - decoder cell is the cell used for decoding
            (Should be part of `tf.nn.rnn_cell`)

        - seq_width shows how many features each input in the sequence has
            (For website fingerprinting this is only 2 (packet_size, incoming))
        - batch_size

        - bidirectional is a boolean value that determines whether the encoder is bidirectional or not
        - reverse is also a boolean value that when if true, reversed the traces for training
    """

    def __init__(self, encoder_cell, decoder_cell, seq_width, batch_size=100, bidirectional=False, reverse=False, saved_graph=None, sess=None, learning_rate=0.0006):
        """
        @param saved_graph is a string, representing the path to the saved graph
        """
        # Constants
        self.PAD = 0
        self.EOS = -1

        self.reverse = reverse
        self.seq_width = seq_width
        self.batch_size = batch_size
        self.learning_rate = learning_rate

        self.bidirectional = bidirectional

        self.encoder_cell = encoder_cell
        self.decoder_cell = decoder_cell

        self._make_graph()

        if saved_graph is not None and sess is not None:
            self.import_from_file(sess, saved_graph)

    def _make_graph(self):
        """
        Construct the graph
        """

        self._init_placeholders()

        self._init_encoder()
        self._init_decoder()

        self._init_train()

    def _init_placeholders(self):
        """
        The main placeholders used for the input data, and output
        """
        # The usual format is: `[self.batch_size, max_sequence_length, self.seq_width]`
        # But we define `max_sequence_length` as None to make it dynamic so we only need to pad
        # each batch to the maximum sequence length
        self.encoder_inputs = tf.placeholder(tf.float32,
            [self.batch_size, None, self.seq_width])

        self.encoder_inputs_length = tf.placeholder(tf.int32, [self.batch_size])

        self.decoder_targets = tf.placeholder(tf.float32,
            [self.batch_size, None, self.seq_width])

    def _init_encoder(self):
        """
        Creates the encoder attributes

        Attributes:
            - encoder_outputs is shaped [max_sequence_length, batch_size, seq_width]
                (since time-major == True)
            - encoder_final_state is shaped [batch_size, encoder_cell.state_size]
        """
        if not self.bidirectional:
            with tf.variable_scope('Encoder') as scope:
                self.encoder_outputs, self.encoder_final_state = tf.nn.dynamic_rnn(
                    cell=self.encoder_cell,
                    dtype=tf.float32,
                    sequence_length=self.encoder_inputs_length,
                    inputs=self.encoder_inputs,
                    time_major=False)
        else:
            ((encoder_fw_outputs,
              encoder_bw_outputs),
             (encoder_fw_final_state,
              encoder_bw_final_state)) = (
                tf.nn.bidirectional_dynamic_rnn(cell_fw=self.encoder_cell,
                                                cell_bw=self.encoder_cell,
                                                inputs=self.encoder_inputs,
                                                sequence_length=self.encoder_inputs_length,
                                                dtype=tf.float32, time_major=False)
                )

            self.encoder_outputs = tf.concat((encoder_fw_outputs, encoder_bw_outputs), 2)

            if isinstance(encoder_fw_final_state, LSTMStateTuple):
                encoder_final_state_c = tf.concat(
                    (encoder_fw_final_state.c, encoder_bw_final_state.c), 1)

                encoder_final_state_h = tf.concat(
                    (encoder_fw_final_state.h, encoder_bw_final_state.h), 1)

                self.encoder_final_state = LSTMStateTuple(
                    c=encoder_final_state_c,
                    h=encoder_final_state_h
                )

            else:
                self.encoder_final_state = tf.concat(
                    (encoder_fw_final_state, encoder_bw_final_state), 1)

    def _init_decoder(self):
        """
        Creates decoder attributes.
        We cannot simply use a dynamic_rnn since we are feeding the outputs of the
        decoder back into the inputs.
        Therefore we use a raw_rnn and emulate a dynamic_rnn with this behavior.
        (https://github.com/tensorflow/tensorflow/blob/master/tensorflow/python/ops/rnn.py)
        """
        # EOS token added
        self.decoder_inputs_length = self.encoder_inputs_length + 1

        def loop_fn_initial(time, cell_output, cell_state, loop_state):
            elements_finished = (time >= self.decoder_inputs_length)

            # EOS token (0 + self.EOS)
            initial_input = tf.zeros([self.batch_size, self.decoder_cell.output_size], dtype=tf.float32) + self.EOS
            initial_cell_state = self.encoder_final_state
            initial_loop_state = None  # we don't need to pass any additional information

            return (elements_finished,
                    initial_input,
                    initial_cell_state,
                    None,  # cell output is dummy here
                    initial_loop_state)

        def loop_fn(time, cell_output, cell_state, loop_state):
            if cell_output is None:  # time == 0
                return loop_fn_initial(time, cell_output, cell_state, loop_state)

            cell_output.set_shape([self.batch_size, self.decoder_cell.output_size])

            emit_output = cell_output

            next_cell_state = cell_state

            elements_finished = (time >= self.decoder_inputs_length)
            finished = tf.reduce_all(elements_finished)

            next_input = tf.cond(
                finished,
                lambda: tf.zeros([self.batch_size, self.decoder_cell.output_size], dtype=tf.float32), # self.PAD
                lambda: cell_output # Use the input from the previous cell
            )

            next_loop_state = None

            return (
                elements_finished,
                next_input,
                next_cell_state,
                emit_output,
                next_loop_state
            )

        decoder_outputs_ta, decoder_final_state, _ = tf.nn.raw_rnn(self.decoder_cell, loop_fn)
        self.decoder_outputs = decoder_outputs_ta.stack()
        self.decoder_outputs = tf.transpose(self.decoder_outputs, [1, 0, 2])

        with tf.variable_scope('DecoderOutputProjection') as scope:
            self.decoder_outputs = self.projection(self.decoder_outputs, self.seq_width, scope)

    def _init_train(self):
        self.loss = tf.reduce_sum(tf.square(self.decoder_targets - self.decoder_outputs))

        # Which optimizer to use? `GradientDescentOptimizer`, `AdamOptimizer` or `RMSProp`?
        self.train_op = tf.train.AdamOptimizer(self.learning_rate).minimize(self.loss)

    def projection(self, inputs, projection_size, scope):
        """
        Projects the input with a known amount of features to a `projection_size amount of features`

        @param inputs is shaped like [time, batch, input_size] or [batch, input_size]
        @param projection_size int32
        @param scope outer variable scope
        """
        input_size = inputs.get_shape()[-1].value

        with tf.variable_scope(scope) as scope:
            W = tf.get_variable(name='W', shape=[input_size, projection_size],
                                dtype=tf.float32)

            b = tf.get_variable(name='b', shape=[projection_size],
                                dtype=tf.float32,
                                initializer=tf.constant_initializer(0, dtype=tf.float32))

        input_shape = tf.unstack(tf.shape(inputs))

        if len(input_shape) == 3:
            time, batch, _ = input_shape  # dynamic parts of shape
            inputs = tf.reshape(inputs, [-1, input_size])

        elif len(input_shape) == 2:
            batch, _depth = input_shape

        else:
            raise ValueError("Weird input shape: {}".format(inputs))

        linear = tf.add(tf.matmul(inputs, W), b)

        if len(input_shape) == 3:
            linear = tf.reshape(linear, [time, batch, projection_size])

        return linear

    def next_batch(self, batches, in_memory, max_time_diff=float("inf")):
        """
        Returns the next batch.

        @param batches an iterator with all of the batches (
            if in_memory == True:
                in batch-major form without padding
            else:
                A list of paths to the files
        )
        @param in_memory is a boolean value
        @param max_time_diff **(should only be defined if `in_memory == False`)**
            specifies what the maximum time different between the first packet in the trace and the last one should be

        @return if in_memory is False, returns a tuple of (dict, [paths], max_length) where paths is a list of paths for each batch
            else it returns a dict for training
        """
        batch = next(batches)
        data_batch = batch

        if not in_memory:
            data_batch = [helpers.read_cell_file(path, max_time_diff=max_time_diff) for path in batch]
            for i, cell in enumerate(data_batch):
                data_batch[i] = [packet[0] * packet[1] for packet in cell]

        data_batch, encoder_input_lengths_ = helpers.pad_traces(data_batch, reverse=self.reverse, seq_width=self.seq_width)
        encoder_inputs_ = data_batch

        decoder_targets_ = helpers.add_EOS(data_batch, encoder_input_lengths_)

        train_dict = {
            self.encoder_inputs: encoder_inputs_,
            self.encoder_inputs_length: encoder_input_lengths_,
            self.decoder_targets: decoder_targets_,
        }

        if not in_memory:
            return (train_dict, batch, max(encoder_input_lengths_))
        return train_dict

    def save(self, sess, file_name):
        """
        Save the model in a file

        @param sess is the session
        @param file_name is the file name without the extension
        """
        saver = tf.train.Saver()
        saver.save(sess, file_name)
        # saver.export_meta_graph(filename=file_name + '.meta')

    def import_from_file(self, sess, file_name):
        """
        Imports the graph from a file

        @param sess is the session
        @param file_name is a string that represents the file name
            without the extension
        """

        # Get the graph
        saver = tf.train.Saver()

        # Restore the variables
        saver.restore(sess, file_name)


def train_on_copy_task(sess, model, data,
                       batch_size=100,
                       max_batches=None,
                       batches_in_epoch=1000,
                       max_time_diff=float("inf"),
                       verbose=False):
    """
    Train the `Seq2SeqModel` on a copy task

    @param sess is a tensorflow session
    @param model is the seq2seq model
    @param data is the data (in batch-major form and not padded or a list of files (depending on `in_memory`))
    """
    batches = helpers.get_batches(data, batch_size=batch_size)

    loss_track = []

    batches_in_data = len(data) // batch_size
    if max_batches is None or batches_in_data < max_batches:
        max_batches = batches_in_data - 1

    try:
        for batch in range(max_batches):
            print("Batch {}/{}".format(batch, max_batches))
            fd, _, length = model.next_batch(batches, False, max_time_diff)
            _, l = sess.run([model.train_op, model.loss], fd)
            loss_track.append(l / length)

            if batch == 0 or batch % batches_in_epoch == 0:
                model.save(sess, 'seq2seq_model')
                helpers.save_object(loss_track, 'loss_track.pkl')

                if verbose:
                    stdout.write('  minibatch loss: {}\n'.format(sess.run(model.loss, fd)))
                    predict_ = sess.run(model.decoder_outputs, fd)
                    for i, (inp, pred) in enumerate(zip(fd[model.encoder_inputs].swapaxes(0, 1), predict_.swapaxes(0, 1))):
                        stdout.write('  sample {}:\n'.format(i + 1))
                        stdout.write('    input     > {}\n'.format(inp))
                        stdout.write('    predicted > {}\n'.format(pred))
                        if i >= 0:
                            break
                    stdout.write('\n')

    except KeyboardInterrupt:
        stdout.write('training interrupted')
        model.save(sess, 'seq2seq_model')
        exit(0)

    model.save(sess, 'seq2seq_model')
    helpers.save_object(loss_track, 'loss_track.pkl')

    return loss_track

def get_vector_representations(sess, model, data, save_dir,
                       batch_size=100,
                       max_batches=None,
                       batches_in_epoch=1000,
                       max_time_diff=float("inf"),
                       extension=".cell"):
    """
    Given a trained model, gets a vector representation for the traces in batch

    @param sess is a tensorflow session
    @param model is the seq2seq model
    @param data is the data (in batch-major form and not padded or a list of files (depending on `in_memory`))
    """
    batches = helpers.get_batches(data, batch_size=batch_size)

    batches_in_data = len(data) // batch_size
    if max_batches is None or batches_in_data < max_batches:
        max_batches = batches_in_data - 1

    try:
        for batch in range(max_batches):
            print("Batch {}/{}".format(batch, max_batches))
            fd, paths, _ = model.next_batch(batches, False, max_time_diff)
            l = sess.run(model.encoder_final_state, fd)

            # Returns a tuple, so we concatenate
            if isinstance(l, LSTMStateTuple):
                l = np.concatenate((l.c, l.h), axis=1)

            file_names = [helpers.extract_filename_from_path(path, extension) for path in paths]

            for file_name, features in zip(file_names, list(l)):
                helpers.write_to_file(features, save_dir, file_name, new_extension=".cellf")

    except KeyboardInterrupt:
        stdout.write('Interrupted')
        exit(0)
\end{lstlisting}

\subsection{Batch-normalized LSTM cell}

\begin{lstlisting}[language=Python]
"""
Implements batch normalized LSTM cells described in https://arxiv.org/pdf/1510.01378.pdf.
"""

from tensorflow.contrib.rnn import LSTMCell, LSTMStateTuple

from tensorflow.python.framework import ops
from tensorflow.python.framework import tensor_shape
from tensorflow.python.framework import tensor_util
from tensorflow.python.ops import array_ops
from tensorflow.python.ops import clip_ops
from tensorflow.python.ops import embedding_ops
from tensorflow.python.ops import init_ops
from tensorflow.python.ops import math_ops
from tensorflow.python.ops import nn_ops
from tensorflow.python.ops import partitioned_variables
from tensorflow.python.ops import random_ops
from tensorflow.python.ops import variable_scope as vs

from tensorflow.python.ops.math_ops import sigmoid
from tensorflow.python.ops.math_ops import tanh
from tensorflow.python.ops.rnn_cell_impl import _RNNCell as RNNCell

from tensorflow.python.platform import tf_logging as logging
from tensorflow.python.util import nest

from tensorflow.contrib.layers import batch_norm

import contextlib

_BIAS_VARIABLE_NAME = "biases"
_WEIGHTS_VARIABLE_NAME = "weights"

@contextlib.contextmanager
def _checked_scope(cell, scope, reuse=None, **kwargs):
  if reuse is not None:
    kwargs["reuse"] = reuse
  with vs.variable_scope(scope, **kwargs) as checking_scope:
    scope_name = checking_scope.name
    if hasattr(cell, "_scope"):
      cell_scope = cell._scope  # pylint: disable=protected-access
      if cell_scope.name != checking_scope.name:
        raise ValueError(
            "Attempt to reuse RNNCell %s with a different variable scope than "
            "its first use.  First use of cell was with scope '%s', this "
            "attempt is with scope '%s'.  Please create a new instance of the "
            "cell if you would like it to use a different set of weights.  "
            "If before you were using: MultiRNNCell([%s(...)] * num_layers), "
            "change to: MultiRNNCell([%s(...) for _ in range(num_layers)]).  "
            "If before you were using the same cell instance as both the "
            "forward and reverse cell of a bidirectional RNN, simply create "
            "two instances (one for forward, one for reverse).  "
            "In May 2017, we will start transitioning this cell's behavior "
            "to use existing stored weights, if any, when it is called "
            "with scope=None (which can lead to silent model degradation, so "
            "this error will remain until then.)"
            % (cell, cell_scope.name, scope_name, type(cell).__name__,
               type(cell).__name__))
    else:
      weights_found = False
      try:
        with vs.variable_scope(checking_scope, reuse=True):
          vs.get_variable(_WEIGHTS_VARIABLE_NAME)
        weights_found = True
      except ValueError:
        pass
      if weights_found and reuse is None:
        raise ValueError(
            "Attempt to have a second RNNCell use the weights of a variable "
            "scope that already has weights: '%s'; and the cell was not "
            "constructed as %s(..., reuse=True).  "
            "To share the weights of an RNNCell, simply "
            "reuse it in your second calculation, or create a new one with "
            "the argument reuse=True." % (scope_name, type(cell).__name__))

    # Everything is OK.  Update the cell's scope and yield it.
    cell._scope = checking_scope  # pylint: disable=protected-access
    yield checking_scope

class BNLSTMCell(LSTMCell):
  """Long short-term memory unit (LSTM) recurrent network cell.
  The default non-peephole implementation is based on:
    http://deeplearning.cs.cmu.edu/pdfs/Hochreiter97_lstm.pdf
  S. Hochreiter and J. Schmidhuber.
  "Long Short-Term Memory". Neural Computation, 9(8):1735-1780, 1997.
  The peephole implementation is based on:
    https://research.google.com/pubs/archive/43905.pdf
  Hasim Sak, Andrew Senior, and Francoise Beaufays.
  "Long short-term memory recurrent neural network architectures for
   large scale acoustic modeling." INTERSPEECH, 2014.
  The class uses optional peep-hole connections, optional cell clipping, and
  an optional projection layer.
  """

  def __init__(self, num_units, input_size=None,
               use_peepholes=False, cell_clip=None,
               initializer=None, num_proj=None, proj_clip=None,
               num_unit_shards=None, num_proj_shards=None,
               forget_bias=1.0, state_is_tuple=True,
               activation=tanh, is_training=True, batch_norm=True):
    """Initialize the parameters for an LSTM cell.
    Args:
      num_units: int, The number of units in the LSTM cell
      input_size: Deprecated and unused.
      use_peepholes: bool, set True to enable diagonal/peephole connections.
      cell_clip: (optional) A float value, if provided the cell state is clipped
        by this value prior to the cell output activation.
      initializer: (optional) The initializer to use for the weight and
        projection matrices.
      num_proj: (optional) int, The output dimensionality for the projection
        matrices.  If None, no projection is performed.
      proj_clip: (optional) A float value.  If `num_proj > 0` and `proj_clip` is
        provided, then the projected values are clipped elementwise to within
        `[-proj_clip, proj_clip]`.
      num_unit_shards: Deprecated, will be removed by Jan. 2017.
        Use a variable_scope partitioner instead.
      num_proj_shards: Deprecated, will be removed by Jan. 2017.
        Use a variable_scope partitioner instead.
      forget_bias: Biases of the forget gate are initialized by default to 1
        in order to reduce the scale of forgetting at the beginning of
        the training.
      state_is_tuple: If True, accepted and returned states are 2-tuples of
        the `c_state` and `m_state`.  If False, they are concatenated
        along the column axis.  This latter behavior will soon be deprecated.
      activation: Activation function of the inner states.
      is_training: Python boolean describing whether or not you are currently
        training. Should only be changed if `batch_norm == True`
      batch_norm: Python boolean that indicated whether or not the cell is
        batch normalized
    """
    self._is_training = is_training
    self._batch_norm = batch_norm

    super().__init__(num_units, input_size=input_size,
                 use_peepholes=use_peepholes, cell_clip=cell_clip,
                 initializer=initializer, num_proj=num_proj, proj_clip=proj_clip,
                 num_unit_shards=num_unit_shards, num_proj_shards=num_proj_shards,
                 forget_bias=forget_bias, state_is_tuple=state_is_tuple,
                 activation=activation)


  def __call__(self, inputs, state, scope=None):
    """Run one step of LSTM.
    Args:
      inputs: input Tensor, 2D, batch x num_units.
      state: if `state_is_tuple` is False, this must be a state Tensor,
        `2-D, batch x state_size`.  If `state_is_tuple` is True, this must be a
        tuple of state Tensors, both `2-D`, with column sizes `c_state` and
        `m_state`.
      scope: VariableScope for the created subgraph; defaults to "lstm_cell".
    Returns:
      A tuple containing:
      - A `2-D, [batch x output_dim]`, Tensor representing the output of the
        LSTM after reading `inputs` when previous state was `state`.
        Here output_dim is:
           num_proj if num_proj was set,
           num_units otherwise.
      - Tensor(s) representing the new state of LSTM after reading `inputs` when
        the previous state was `state`.  Same type and shape(s) as `state`.
    Raises:
      ValueError: If input size cannot be inferred from inputs via
        static shape inference.
    """
    num_proj = self._num_units if self._num_proj is None else self._num_proj

    if self._state_is_tuple:
      (c_prev, m_prev) = state
    else:
      c_prev = array_ops.slice(state, [0, 0], [-1, self._num_units])
      m_prev = array_ops.slice(state, [0, self._num_units], [-1, num_proj])

    dtype = inputs.dtype
    input_size = inputs.get_shape().with_rank(2)[1]
    if input_size.value is None:
      raise ValueError("Could not infer input size from inputs.get_shape()[-1]")
    with _checked_scope(self, scope or "lstm_cell",
                        initializer=self._initializer) as unit_scope:
      if self._num_unit_shards is not None:
        unit_scope.set_partitioner(
            partitioned_variables.fixed_size_partitioner(
                self._num_unit_shards))
      # i = input_gate, j = new_input, f = forget_gate, o = output_gate
      lstm_matrix = _linear([inputs, m_prev], 4 * self._num_units, bias=True)
      i, j, f, o = array_ops.split(
          value=lstm_matrix, num_or_size_splits=4, axis=1)
      # Diagonal connections
      if self._use_peepholes:
        with vs.variable_scope(unit_scope) as projection_scope:
          if self._num_unit_shards is not None:
            projection_scope.set_partitioner(None)
          w_f_diag = vs.get_variable(
              "w_f_diag", shape=[self._num_units], dtype=dtype)
          w_i_diag = vs.get_variable(
              "w_i_diag", shape=[self._num_units], dtype=dtype)
          w_o_diag = vs.get_variable(
              "w_o_diag", shape=[self._num_units], dtype=dtype)

      if self._use_peepholes:
        res = (sigmoid(f + self._forget_bias + w_f_diag * c_prev) * c_prev +
             sigmoid(i + w_i_diag * c_prev) * self._activation(j))
        if self._batch_norm:
          c = batch_norm(res,
                         center=True, scale=True,
                         is_training=self._is_training,
                         scope='bn')
        else:
          c = res
      else:
        res = (sigmoid(f + self._forget_bias) * c_prev + sigmoid(i) *
             self._activation(j))
        if self._batch_norm:
          c = batch_norm(res,
                         center=True, scale=True,
                         is_training=self._is_training,
                         scope='bn')
        else:
          c = res

      if self._cell_clip is not None:
        # pylint: disable=invalid-unary-operand-type
        c = clip_ops.clip_by_value(c, -self._cell_clip, self._cell_clip)
        # pylint: enable=invalid-unary-operand-type
      if self._use_peepholes:
        m = sigmoid(o + w_o_diag * c) * self._activation(c)
      else:
        m = sigmoid(o) * self._activation(c)

      if self._num_proj is not None:
        with vs.variable_scope("projection") as proj_scope:
          if self._num_proj_shards is not None:
            proj_scope.set_partitioner(
                partitioned_variables.fixed_size_partitioner(
                    self._num_proj_shards))
          m = _linear(m, self._num_proj, bias=False)

        if self._proj_clip is not None:
          # pylint: disable=invalid-unary-operand-type
          m = clip_ops.clip_by_value(m, -self._proj_clip, self._proj_clip)
          # pylint: enable=invalid-unary-operand-type

    new_state = (LSTMStateTuple(c, m) if self._state_is_tuple else
                 array_ops.concat([c, m], 1))
    return m, new_state

def _linear(args, output_size, bias, bias_start=0.0):
  """Linear map: sum_i(args[i] * W[i]), where W[i] is a variable.
  Args:
    args: a 2D Tensor or a list of 2D, batch x n, Tensors.
    output_size: int, second dimension of W[i].
    bias: boolean, whether to add a bias term or not.
    bias_start: starting value to initialize the bias; 0 by default.
  Returns:
    A 2D Tensor with shape [batch x output_size] equal to
    sum_i(args[i] * W[i]), where W[i]s are newly created matrices.
  Raises:
    ValueError: if some of the arguments has unspecified or wrong shape.
  """
  if args is None or (nest.is_sequence(args) and not args):
    raise ValueError("`args` must be specified")
  if not nest.is_sequence(args):
    args = [args]

  # Calculate the total size of arguments on dimension 1.
  total_arg_size = 0
  shapes = [a.get_shape() for a in args]
  for shape in shapes:
    if shape.ndims != 2:
      raise ValueError("linear is expecting 2D arguments: %s" % shapes)
    if shape[1].value is None:
      raise ValueError("linear expects shape[1] to be provided for shape %s, "
                       "but saw %s" % (shape, shape[1]))
    else:
      total_arg_size += shape[1].value

  dtype = [a.dtype for a in args][0]

  # Now the computation.
  scope = vs.get_variable_scope()
  with vs.variable_scope(scope) as outer_scope:
    weights = vs.get_variable(
        _WEIGHTS_VARIABLE_NAME, [total_arg_size, output_size], dtype=dtype)
    if len(args) == 1:
      res = math_ops.matmul(args[0], weights)
    else:
      res = math_ops.matmul(array_ops.concat(args, 1), weights)
    if not bias:
      return res
    with vs.variable_scope(outer_scope) as inner_scope:
      inner_scope.set_partitioner(None)
      biases = vs.get_variable(
          _BIAS_VARIABLE_NAME, [output_size],
          dtype=dtype,
          initializer=init_ops.constant_initializer(bias_start, dtype=dtype))
    return nn_ops.bias_add(res, biases)
\end{lstlisting}

\subsection{Batch-normalized GRU cell}
\begin{lstlisting}[language=Python]
"""
Implements a batch normalized GRU cell.
Unfortunately, there aren't any research papers that examine how to exactly implement this.

But most of it is based on https://arxiv.org/pdf/1510.01378.pdf
"""

from tensorflow.contrib.rnn import GRUCell

from tensorflow.python.framework import ops
from tensorflow.python.framework import tensor_shape
from tensorflow.python.framework import tensor_util
from tensorflow.python.ops import array_ops
from tensorflow.python.ops import clip_ops
from tensorflow.python.ops import embedding_ops
from tensorflow.python.ops import init_ops
from tensorflow.python.ops import math_ops
from tensorflow.python.ops import nn_ops
from tensorflow.python.ops import partitioned_variables
from tensorflow.python.ops import random_ops
from tensorflow.python.ops import variable_scope as vs

from tensorflow.python.ops.math_ops import sigmoid
from tensorflow.python.ops.math_ops import tanh
from tensorflow.python.ops.rnn_cell_impl import _RNNCell as RNNCell

from tensorflow.python.platform import tf_logging as logging
from tensorflow.python.util import nest

from tensorflow.contrib.layers import batch_norm

import contextlib

_BIAS_VARIABLE_NAME = "biases"
_WEIGHTS_VARIABLE_NAME = "weights"

@contextlib.contextmanager
def _checked_scope(cell, scope, reuse=None, **kwargs):
  if reuse is not None:
    kwargs["reuse"] = reuse
  with vs.variable_scope(scope, **kwargs) as checking_scope:
    scope_name = checking_scope.name
    if hasattr(cell, "_scope"):
      cell_scope = cell._scope  # pylint: disable=protected-access
      if cell_scope.name != checking_scope.name:
        raise ValueError(
            "Attempt to reuse RNNCell %s with a different variable scope than "
            "its first use.  First use of cell was with scope '%s', this "
            "attempt is with scope '%s'.  Please create a new instance of the "
            "cell if you would like it to use a different set of weights.  "
            "If before you were using: MultiRNNCell([%s(...)] * num_layers), "
            "change to: MultiRNNCell([%s(...) for _ in range(num_layers)]).  "
            "If before you were using the same cell instance as both the "
            "forward and reverse cell of a bidirectional RNN, simply create "
            "two instances (one for forward, one for reverse).  "
            "In May 2017, we will start transitioning this cell's behavior "
            "to use existing stored weights, if any, when it is called "
            "with scope=None (which can lead to silent model degradation, so "
            "this error will remain until then.)"
            % (cell, cell_scope.name, scope_name, type(cell).__name__,
               type(cell).__name__))
    else:
      weights_found = False
      try:
        with vs.variable_scope(checking_scope, reuse=True):
          vs.get_variable(_WEIGHTS_VARIABLE_NAME)
        weights_found = True
      except ValueError:
        pass
      if weights_found and reuse is None:
        raise ValueError(
            "Attempt to have a second RNNCell use the weights of a variable "
            "scope that already has weights: '%s'; and the cell was not "
            "constructed as %s(..., reuse=True).  "
            "To share the weights of an RNNCell, simply "
            "reuse it in your second calculation, or create a new one with "
            "the argument reuse=True." % (scope_name, type(cell).__name__))

    # Everything is OK.  Update the cell's scope and yield it.
    cell._scope = checking_scope  # pylint: disable=protected-access
    yield checking_scope


class BNGRUCell(GRUCell):
  """Gated Recurrent Unit cell (cf. http://arxiv.org/abs/1406.1078)."""

  def __init__(self, num_units, input_size=None, activation=tanh, is_training=True, batch_norm=True):
    self._is_training = is_training
    self._batch_norm = batch_norm

    super().__init__(num_units, input_size, activation)

  def __call__(self, inputs, state, scope=None):
    """Gated recurrent unit (GRU) with nunits cells."""
    with _checked_scope(self, scope or "gru_cell"):
      with vs.variable_scope("gates"):  # Reset gate and update gate.
        # We start with bias of 1.0 to not reset and not update.
        value = sigmoid(_linear(
          [inputs, state], 2 * self._num_units, True, 1.0))
        r, u = array_ops.split(
            value=value,
            num_or_size_splits=2,
            axis=1)
      with vs.variable_scope("candidate"):
        res = self._activation(_linear([inputs, r * state],
                                     self._num_units, True))

        if self._batch_norm:
          c = batch_norm(res,
                         center=True, scale=True,
                         is_training=self._is_training,
                         scope='bn1')
        else:
          c = res

      new_h = u * state + (1 - u) * c
    return new_h, new_h

def _linear(args, output_size, bias, bias_start=0.0):
  """Linear map: sum_i(args[i] * W[i]), where W[i] is a variable.
  Args:
    args: a 2D Tensor or a list of 2D, batch x n, Tensors.
    output_size: int, second dimension of W[i].
    bias: boolean, whether to add a bias term or not.
    bias_start: starting value to initialize the bias; 0 by default.
  Returns:
    A 2D Tensor with shape [batch x output_size] equal to
    sum_i(args[i] * W[i]), where W[i]s are newly created matrices.
  Raises:
    ValueError: if some of the arguments has unspecified or wrong shape.
  """
  if args is None or (nest.is_sequence(args) and not args):
    raise ValueError("`args` must be specified")
  if not nest.is_sequence(args):
    args = [args]

  # Calculate the total size of arguments on dimension 1.
  total_arg_size = 0
  shapes = [a.get_shape() for a in args]
  for shape in shapes:
    if shape.ndims != 2:
      raise ValueError("linear is expecting 2D arguments: %s" % shapes)
    if shape[1].value is None:
      raise ValueError("linear expects shape[1] to be provided for shape %s, "
                       "but saw %s" % (shape, shape[1]))
    else:
      total_arg_size += shape[1].value

  dtype = [a.dtype for a in args][0]

  # Now the computation.
  scope = vs.get_variable_scope()
  with vs.variable_scope(scope) as outer_scope:
    weights = vs.get_variable(
        _WEIGHTS_VARIABLE_NAME, [total_arg_size, output_size], dtype=dtype)
    if len(args) == 1:
      res = math_ops.matmul(args[0], weights)
    else:
      res = math_ops.matmul(array_ops.concat(args, 1), weights)
    if not bias:
      return res
    with vs.variable_scope(outer_scope) as inner_scope:
      inner_scope.set_partitioner(None)
      biases = vs.get_variable(
          _BIAS_VARIABLE_NAME, [output_size],
          dtype=dtype,
          initializer=init_ops.constant_initializer(bias_start, dtype=dtype))
    return nn_ops.bias_add(res, biases)
\end{lstlisting}

\endgroup

\end{multicols}
\end{landscape}
