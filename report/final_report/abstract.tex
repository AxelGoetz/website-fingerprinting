\begin{abstract}

Low-latency anonymity networks such as Tor allow people to anonymously browse the web by obscuring both the content and meta-data of their communications.
However, there are several attacks that allow an avdersary to break the privacy expected by their users.
At the time of writing, one of the most popular attacks against Tor are called \textit{website fingerprinting}, which rely a local, passive eavesdropper performing pattern analysis on the encrypted data.
In prior works, adversaries extract \textit{fingerprints} from the traffic data, which are then used to classify web pages accordingly.
But in order to do so these works rely on a time-consuming, laborious feature extraction process.

Here, we attempt to automate this process by leveraging several unsupervised deep learning techniques.
More specifically, a \textit{stacked autoencoder} and a \textit{sequence-to-sequence model}.
Next, we also propose a technique to evaluate the performance of these models and use this same method to compare them to current state-of-the-art attacks.

We demonstrate that in an open-world scenario, the sequence-to-sequence model seems to get the highest performance with a $93 \%$ and a $39 \%$ accuracy for a binary and a multiclass classification task, respectively.
On top of this, we also show that both deep learning models can be used to extract fingerprints from data that was recorded under different circumstances and therefore they do not to be retrained after a short period of time, unlike other state-of-the-art attacks.

Although our approach currently does not outperform previous attacks, it shows us that it is in fact possible to automate the feature extraction process without the need of any domain-specific knowledge.

\end{abstract}
