\begin{abstract}

Anonymity networks such as Tor enable their users to anonymously browse the Internet.
But there are several attacks that allow adversaries to identify users, one of these is called \textit{website fingerprinting}, which relies a local, passive eavesdropper who performs pattern analysis on the encrypted data to classify web pages.
In most prior works, adversaries extract \textit{fingerprints} by relying on a time-consuming, laborious feature extraction process.
Here, we present several deep-learning techniques (\textit{stacked autoencoder} and a \textit{sequence-to-sequence model}) that can be used to automate this process, propose a technique to evaluate their performance and use this technique to compare our models to current state-of-the-art attacks.

We find that the sequence-to-sequence models seems to get the highest performance with a $93 \%$ maximum for a binary classification task and a $39 \%$ accuracy in the multiclass classification both within an open-world scenario.
On top of this, we also show that both deep learning models are robust and can be used to extract fingerprints from data that was recorded under different circumstances.

Although our approach currently does not outperform the state-of-the-art attacks, it shows us that it is in fact possible to automate the feature extraction process without the need of any domain-specific knowledge.
\end{abstract}
