\section{Threat Model and Requirements}

In the following section, we further describe the adversary models from which we deduce a list of requirements.

\subsection{Threat Model}

We consider an adversary, who wants to perform a website fingerprinting attack against Tor.
Tor, specifically, since it has become one of the most widely used internet privacy tools.
As specified in figure \ref{fig:threat_model}, the adversary is a \textit{local eavesdropper}.
Hence, the attacker passively collects encrypted web traffic between the client and the first tor node, or the \textit{entry guard}.
This is achieved by either monitoring the link itself or a compromised entry node.
Next, it performs an analysis on that data to classify which specific web pages the client is visiting.

This analysis can be performed with several different goals.
The first one is to identify whether or not a user visits a web page from a set of monitored web pages.
Thus the attack is essentially a \textit{binary classification problem}, where you label a web page as \textit{monitored} or \textit{unmonitored}.
Or the adversary might want to know which specific web pages a user visits, or a \textit{multiclass classification problem}.

Within this adversary model we do make various assumptions.
First of all, the adversary is not interested in blocking Tor traffic nor in modifying any traffic.
Next, the adversary is able to replicate the conditions under which the user browses the internet such as download speeds, OS and TBB.
On top of this, the adversary can also determine the beginning and the end of a user session on a web page and that the attacker has enough resources to collect traffic and train a deep learning model.
Finally, we also make the adversary is unable to decrypt the traffic and thus only has access to metadata such as traffic direction, length and timing of packets.

This model described above is consistent with previously discussed works in section \ref{sec:related-work}.

\subsection{Problem Statement}

At the time of writing, most WF works make use of machine learning techniques that require a fixed-length input.
Even though a traffic trace consists of a variable-length sequence of packets.
Therefore, these works often rely on a laborious, time-consuming process to extract fixed-length representations, or \textit{fingerprints}.
But there is no guarantee that these fingerprints are the most appropriate ones.
On top of that, the previously mentioned process often requires domain-specific knowledge, making the attack even more difficult.
Thus, here we investigate the use of automatic feature generation techniques to extract features automatically, without the need for any domain-specific knowledge.


\subsection{Requirements}

This works aims to provide the code to perform the feature generation process.
Next, it also and to evaluate its effectiveness.
Next, it also provides the necessary documentation to re
